\documentclass[10pt]{article}
\usepackage{amsmath}
\usepackage{bm}
%\usepackage{bbm}
\usepackage{mathrsfs}
\usepackage{graphicx}
%\usepackage{wrapfig}
\usepackage{subcaption}
\usepackage{epsfig}
\usepackage{amsfonts}
\usepackage{amssymb}
\usepackage{amsmath}
\usepackage{graphicx}
\usepackage{psfrag}
\usepackage{cleveref}
\usepackage{xcolor, soul}
\sethlcolor{yellow}
\newcommand{\sun}{\ensuremath{\odot}} % sun symbol is \sun
\let\vaccent=\v % rename builtin command \v{} to \vaccent{}
\renewcommand{\v}[1]{\ensuremath{\mathbf{#1}}} % for vectors
\newcommand{\gv}[1]{\ensuremath{\mbox{\boldmath$ #1 $}}} 
\newcommand{\grad}[1]{\gv{\nabla} #1}
\renewcommand{\baselinestretch}{1.2}
\jot 5mm
\graphicspath{{./figures/}}
%text dimensions
\textwidth 6.5 in
\oddsidemargin .2 in
\topmargin -0.2 in
\textheight 8.5 in
\headheight 0.2in
\overfullrule = 0pt
\pagestyle{plain}
\def\newpar{\par\vskip 0.5cm}
\begin{document}
%
%----------------------------------------------------------------------
%        Define symbols
%----------------------------------------------------------------------
%
\def\iso{\mathbbm{1}}
\def\half{{\textstyle{1 \over 2}}}
\def\third{{\textstyle{1 \over 3}}}
\def\fourth{{\textstyle{{1 \over 4}}}}
\def\twothird{{\textstyle {{2 \over 3}}}}
\def\ndim{{n_{\rm dim}}}
\def\nint{n_{\rm int}}
\def\lint{l_{\rm int}}
\def\nel{n_{\rm el}}
\def\nf{n_{\rm f}}
\def\DIV {\hbox{\af div}}
\def\GRAD{\hbox{\af Grad}}
\def\sym{\mathop{\rm sym}\nolimits}
\def\tr{\mathop{\rm tr}\nolimits}
\def\dev{\mathop{\rm dev}\nolimits}
\def\Dev{\mathop{\rm Dev}\nolimits}
\def\DEV{\mathop {\rm DEV}\nolimits}
\def\bfb {{\bi b}}
\def\Bnabla{\nabla}
\def\bG{{\bi G}}
\def\jmpdelu{{\lbrack\!\lbrack \Delta u\rbrack\!\rbrack}}
\def\jmpudot{{\lbrack\!\lbrack\dot u\rbrack\!\rbrack}}
\def\jmpu{{\lbrack\!\lbrack u\rbrack\!\rbrack}}
\def\jmphi{{\lbrack\!\lbrack\varphi\rbrack\!\rbrack}}
\def\ljmp{{\lbrack\!\lbrack}}
\def\rjmp{{\rbrack\!\rbrack}}
\def\sign{{\rm sign}}
\def\nn{{n+1}}
\def\na{{n+\vartheta}}
\def\nna{{n+(1-\vartheta)}}
\def\nt{{n+{1\over 2}}}
\def\nb{{n+\beta}}
\def\nbb{{n+(1-\beta)}}
%---------------------------------------------------------
%               Bold Face Math Characters:
%               All In Format: \B***** .
%---------------------------------------------------------
\def\bOne{\mbox{\boldmath$1$}}
\def\BGamma{\mbox{\boldmath$\Gamma$}}
\def\BDelta{\mbox{\boldmath$\Delta$}}
\def\BTheta{\mbox{\boldmath$\Theta$}}
\def\BLambda{\mbox{\boldmath$\Lambda$}}
\def\BXi{\mbox{\boldmath$\Xi$}}
\def\BPi{\mbox{\boldmath$\Pi$}}
\def\BSigma{\mbox{\boldmath$\Sigma$}}
\def\BUpsilon{\mbox{\boldmath$\Upsilon$}}
\def\BPhi{\mbox{\boldmath$\Phi$}}
\def\BPsi{\mbox{\boldmath$\Psi$}}
\def\BOmega{\mbox{\boldmath$\Omega$}}
\def\Balpha{\mbox{\boldmath$\alpha$}}
\def\Bbeta{\mbox{\boldmath$\beta$}}
\def\Bgamma{\mbox{\boldmath$\gamma$}}
\def\Bdelta{\mbox{\boldmath$\delta$}}
\def\Bepsilon{\mbox{\boldmath$\epsilon$}}
\def\Bzeta{\mbox{\boldmath$\zeta$}}
\def\Beta{\mbox{\boldmath$\eta$}}
\def\Btheta{\mbox{\boldmath$\theta$}}
\def\Biota{\mbox{\boldmath$\iota$}}
\def\Bkappa{\mbox{\boldmath$\kappa$}}
\def\Blambda{\mbox{\boldmath$\lambda$}}
\def\Bmu{\mbox{\boldmath$\mu$}}
\def\Bnu{\mbox{\boldmath$\nu$}}
\def\Bxi{\mbox{\boldmath$\xi$}}
\def\Bpi{\mbox{\boldmath$\pi$}}
\def\Brho{\mbox{\boldmath$\rho$}}
\def\Bsigma{\mbox{\boldmath$\sigma$}}
\def\Btau{\mbox{\boldmath$\tau$}}
\def\Bupsilon{\mbox{\boldmath$\upsilon$}}
\def\Bphi{\mbox{\boldmath$\phi$}}
\def\Bchi{\mbox{\boldmath$\chi$}}
\def\Bpsi{\mbox{\boldmath$\psi$}}
\def\Bomega{\mbox{\boldmath$\omega$}}
\def\Bvarepsilon{\mbox{\boldmath$\varepsilon$}}
\def\Bvartheta{\mbox{\boldmath$\vartheta$}}
\def\Bvarpi{\mbox{\boldmath$\varpi$}}
\def\Bvarrho{\mbox{\boldmath$\varrho$}}
\def\Bvarsigma{\mbox{\boldmath$\varsigma$}}
\def\Bvarphi{\mbox{\boldmath$\varphi$}}
\def\bone{\mathbf{1}}
\def\bzero{\mathbf{0}}
%---------------------------------------------------------
%               Bold Face Math Italic:
%               All In Format: \b* .
%---------------------------------------------------------
\def\bA{\mbox{\boldmath$ A$}}
\def\bB{\mbox{\boldmath$ B$}}
\def\bC{\mbox{\boldmath$ C$}}
\def\bD{\mbox{\boldmath$ D$}}
\def\bE{\mbox{\boldmath$ E$}}
\def\bF{\mbox{\boldmath$ F$}}
\def\bG{\mbox{\boldmath$ G$}}
\def\bH{\mbox{\boldmath$ H$}}
\def\bI{\mbox{\boldmath$ I$}}
\def\bJ{\mbox{\boldmath$ J$}}
\def\bK{\mbox{\boldmath$ K$}}
\def\bL{\mbox{\boldmath$ L$}}
\def\bM{\mbox{\boldmath$ M$}}
\def\bN{\mbox{\boldmath$ N$}}
\def\bO{\mbox{\boldmath$ O$}}
\def\bP{\mbox{\boldmath$ P$}}
\def\bQ{\mbox{\boldmath$ Q$}}
\def\bR{\mbox{\boldmath$ R$}}
\def\bS{\mbox{\boldmath$ S$}}
\def\bT{\mbox{\boldmath$ T$}}
\def\bU{\mbox{\boldmath$ U$}}
\def\bV{\mbox{\boldmath$ V$}}
\def\bW{\mbox{\boldmath$ W$}}
\def\bX{\mbox{\boldmath$ X$}}
\def\bY{\mbox{\boldmath$ Y$}}
\def\bZ{\mbox{\boldmath$ Z$}}
\def\ba{\mbox{\boldmath$ a$}}
\def\bb{\mbox{\boldmath$ b$}}
\def\bc{\mbox{\boldmath$ c$}}
\def\bd{\mbox{\boldmath$ d$}}
\def\be{\mbox{\boldmath$ e$}}
\def\bff{\mbox{\boldmath$ f$}}
\def\bg{\mbox{\boldmath$ g$}}
\def\bh{\mbox{\boldmath$ h$}}
\def\bi{\mbox{\boldmath$ i$}}
\def\bj{\mbox{\boldmath$ j$}}
\def\bk{\mbox{\boldmath$ k$}}
\def\bl{\mbox{\boldmath$ l$}}
\def\bm{\mbox{\boldmath$ m$}}
\def\bn{\mbox{\boldmath$ n$}}
\def\bo{\mbox{\boldmath$ o$}}
\def\bp{\mbox{\boldmath$ p$}}
\def\bq{\mbox{\boldmath$ q$}}
\def\br{\mbox{\boldmath$ r$}}
\def\bs{\mbox{\boldmath$ s$}}
\def\bt{\mbox{\boldmath$ t$}}
\def\bu{\mbox{\boldmath$ u$}}
\def\bv{\mbox{\boldmath$ v$}}
\def\bw{\mbox{\boldmath$ w$}}
\def\bx{\mbox{\boldmath$ x$}}
\def\by{\mbox{\boldmath$ y$}}
\def\bz{\mbox{\boldmath$ z$}}
%*********************************
%Start main paper
%*********************************
\centerline{\Large{\bf PRISMS-PF Application Formulation:}}
\smallskip
\centerline{\Large{\bf alloySolidification\_uniformT}}
\bigskip
This example application implements a simple model to simulate solidification of a binary alloy A-B in the dilute limit with component B acting as a solute in a matrix of A. The implemented model was introduced by Karma.~\cite{Karma2001} in 2001. In this model, latent heat is assumed to diffuse much faster than impurities and, therefore, the temperature field is considered to be fixed by external conditions. In contrast to {\it alloySolidification}, this application considers solidification under uniform temperature.  In the default settings of the application, the simulation starts with a circular solid in the corner of a square system. The evolution of the system is calculated for a supersaturation value, $\Omega$, that remains constant throughout the simulation. As this seed grows, three variables are tracked, an order parameter, $\phi$, that denotes whether the material a liquid ($\phi=-1$) or solid ($\phi=1$), the solute concentration, $c$, and an auxiliary term, $\xi$. 

\section{Model}
\hl{Pending: Deriving the following equations from the free energy}\\
\\
The coupled governing equations for the $\phi$ and $c$ are
\begin{equation}
\tau(\theta)\frac{\partial \phi}{\partial t} = \xi(\phi,c),
\end{equation}
where
\begin{equation}
\xi(\phi,c) = -f'(\phi) - \frac{\lambda}{1-k} g'(\phi)(e^u - 1) + \nabla \cdot [W(\theta)^2\nabla \phi]-\frac{\partial}{\partial x} \left[W(\theta)W'(\theta)\frac{\partial \phi}{\partial y}\right] + \frac{\partial}{\partial y} \left[W(\theta)W'(\theta)\frac{\partial \phi}{\partial x}\right]
\end{equation}
and
\begin{equation}
\frac{\partial c}{\partial t} = \nabla \cdot \vec{j},
\end{equation}
where
\begin{equation}
\vec{j}=-D c q(\phi)\nabla(u) - aWc_l^0(1-k)e^u\frac{\partial \phi}{\partial t}\frac{\nabla \phi}{|\nabla \phi|}
\end{equation}
\hl{Pending: defining $u$, $\theta$, $W(\theta)$, and the other variables and constants}

\section{Model Constants}
\hl{Pending: Update this section}\\
$\epsilon$: Strength of the anisotropay ($\epsilon_4$ for a solid with fourfold anisotropy)\\
$k$: Partition coefficient \\
$c_0$: Initial liquid concentration ($c_\infty$)\\
$\lambda$: Coupling constant (setting this value fixes the interface width) \\
$\tilde{D}$: Dimensionless solute diffusivity in the liquid phase. \\
$\tilde{V}_p$: Dimensionless steady-state velocity of the tip. \\
$\tilde{l}_T$: Dimensionless thermal length. \\
$U_0$: Initial constitutional undercooling of the system ($U_0=-1$ sets the concentration of the liquid as $c_\infty$ and of the solid as $kc_\infty$)\\
$U_{\text{off}}$: Undercooling offset that determines the initial temperature at the interface ($U_{\text{off}}=0$ sets it to the solidus temperature,  $U_{\text{off}}=1$ sets it to the liquidus temperature).\\
$\tilde{y}_0$: Initial solid-liquid interface position relative to the bottom of the system ($\tilde{y}=0$) \\
\section{Time Discretization}
Considering forward Euler explicit time stepping, we have the time discretized kinetics equations:
\begin{equation}
\phi^{n+1}=\phi^{n} + \frac{\xi^n}{\tau^n}\Delta t,
\end{equation}
\begin{equation}
c^{n+1}=c^{n}+\Delta t \left \{ D c^n q(\phi^n)\nabla(u^n) + aWc_l^0(1-k)(e^u)^n \left(\frac{\partial \phi}{\partial t}\right)^n\frac{\nabla \phi^n}{|\nabla \phi^n|} \right \} 
\end{equation}
and
\begin{equation}
\begin{split}
\xi(\phi,c) = & -f'(\phi^n) - \frac{\lambda}{1-k} g'(\phi^n)[(e^u)^n - 1]\\
& + \nabla \cdot [W(\theta^n)^2\nabla \phi^n]-\frac{\partial}{\partial x} \left[W(\theta^n)W'(\theta^n)\frac{\partial \phi^n}{\partial y}\right] + \frac{\partial}{\partial y} \left[W(\theta^n)W'(\theta^n)\frac{\partial \phi^n}{\partial x}\right].
\end{split}
\end{equation}
\section{Weak Formulation}
The weak form of the time-discretized equations for $\phi$,  $c$, and, $\xi$ is
\begin{equation}
\int_{\Omega}   \omega  \phi^{n+1}  ~dV = \int_{\Omega}   \omega \underbrace{\left(\phi^n + \frac{ \xi^n}{\tau(\theta^n)}\Delta t\right)}_{r_{\phi}} ~dV,
\end{equation}
\begin{equation}
\begin{split}
\int_{\Omega}   \omega  c^{n+1}  ~dV =& 
\int_{\Omega} \omega \underbrace{ c^{n} }_{r_c}~dV\\
&+\int_{\Omega}  \nabla  \omega  \cdot \underbrace{\left[-\Delta t D\left(q(\phi^n)\nabla c^n + \frac{(1-k)q(\phi^n)c^n\nabla(\phi^n)}{1+k-(1-k)\phi^n}\right)-\Delta t aWc_l^0(1-k)(e^u)^n \left(\frac{\partial \phi}{\partial t}\right)^n\frac{\nabla \phi^n}{|\nabla \phi^n|} \right]}_{r_cx}~dV,
\end{split}
\end{equation}
\newpage
and
\begin{equation}
\int_{\Omega}   \omega \xi^{n+1} ~dV =\int_{\Omega} \omega r_\xi ~dV + \int_{\Omega} \nabla \omega r_{\xi x} ~dV
\end{equation}
where
\begin{equation}
r_\xi= -f'(\phi^n) - \frac{\lambda}{1-k} g'(\phi^n)[(e^u)^n - 1]
\end{equation}
and 
\begin{equation}
\begin{split}
r_{\xi x}= &-\left[W(\theta^n)^2\frac{\partial \phi^n}{\partial x}-W(\theta^n)W'(\theta^n)\frac{\partial \phi^n}{\partial y}\right]\hat{x}\\
&-\left[W(\theta^n)^2\frac{\partial \phi^n}{\partial y}+W(\theta^n)W'(\theta^n)\frac{\partial \phi^n}{\partial x}\right]\hat{y}
\end{split}
\end{equation}
%\vskip 0.25in
%The above values of $r_{\phi}$, $r_{c}$, $r_{cx}$,  $r_{\xi}$ and  $r_{\xi x}$ are used to define the residuals in the following parameters file: \\
%\textit{applications/alloySolification_uniformT/equations.cc}

\begin{thebibliography}{plain}
    \bibitem{Karma2001} A. Karma, Quantitative phase-field model of alloy solidification, \emph{Phys. Rev. Lett.} {\bf 87}, 115701 (2001).
\end{thebibliography}

\end{document} 